\documentclass[letterpaper]{article} 
% Required Packages 
\usepackage{aaai} 
\usepackage{times} 
\usepackage{helvet} 
\usepackage{courier} 
\setlength{\pdfpagewidth}{8.5in} 
\setlength{\pdfpageheight}{11in} 

%%%%%%%%%%
% PDFINFO for PDFLATEX
% Uncomment and complete the following for metadata (your paper must compile with PDFLATEX)
\pdfinfo{
/Title (Input Your Paper Title Here)
/Author (Sarah Jundt, Shannon Lubetich)
/Keywords (Input your paper�s keywords in this optional area)
}


%%%%%%%%%%
% Section Numbers
% Uncomment if you want to use section numbers % and change the 0 to a 1 or 2
% \setcounter{secnumdepth}{0} %%%%%%%%%%
% Title, Author, and Address Information 
\title{Sponsor group assignment at Pomona College as a Constraint Satisfaction Problem}
\author{Sarah Jundt \and Shannon Lubetich\\
Pomona College\\}
%%%%%%%%%%
% Body of Paper Begins
\begin{document}
\maketitle


\section{Motivation}
The first year of college can be a life-changing experience. This is often greatly affected by the people students meet, and especially the people they live with. At Pomona College, sponsor groups are an essential aspect of first-year culture. Roommates and sponsor groups are currently paired by Head Sponsors, who have to sift through hundreds of applications. If roommate pairs and sponsor groups could be generated by a computer, less manual work would be required and less time would be spent.

\section{Introduction}
Sponsor groups are created by first considering all individual students and matching them with a peer, thus creating roommate pairs. Next, each pair is matched with four to nine other pairs in order to form a sponsor group. Our project will focus on mimicking this process. We will determine reasonable constraints on roommates based on preferences they submitted on housing applications and use these to generate roommate pairs. We will then use joint preferences of roommate pairs to place these students within sponsor groups.

\section{Section Description}
The problem of placing students into roommate pairs and sponsor groups will be configured as two separate constraint satisfaction problems. Students will first be placed into roommate pairs and will then be placed into sponsor groups. Each of these placements will use backtracking and each will return an assignment satisfying specified constraints. 

The constraints for each problem will be based on maximization of similarity between the preferences of students within a pair of sponsor group. The constraint for roommate pairs will be that paired students do not have any preference rankings that differ by more than some constant number. The constant will be determined by experimentation and will be the highest integer for which a successful assignment of roommate pairs can be found.

Backtracking will be used to solve these problems. Within this algorithm, we will select the next student or roommate pair to be assigned using the least-remaining values heuristic in order to reduce the running-time of backtracking. We will also use the AC-3 algorithm as the inference subroutine to guarantee arc-consistency with the current assignment after each addition to the assignment. Once a complete assignment is found of all students into roommate pairs and all roommate pairs into sponsor groups, this assignment is returned as a success.

\section{Results}
We aim to obtain groupings of students into sponsor groups after placing individual students in roommate pairs. 
The success of the constraint-satisfaction problem will be evaluated using K-means clustering. The students will be clustered into 30 sponsor groups based on their preferences. Those clusterings will then be evaluated using purity, where the label associated with each student is the sponsor group number assigned to them by the constraint satisfaction results. A higher purity will indicate that our constraint satisfaction has placed students in sponsor groups with other similar students.

\section{Bibliography}

S. Gravel, V. Elser. �Divide and concur: A general approach to constraint satisfaction.� Physical Review E 78, 036706 (2008). American Physical Society.  \\

\noindent A.K. Mackworth, E.C. Freuder. �The complexity of some polynomial network consistency algorithms for constraint satisfaction problems.� Artificial Intelligence, 25 (1985), pp. 65-74


%%%%%%%%%%
% References and End of Paper 
\bibliography{Bibliography-File} 
\bibliographystyle{aaai}
\end{document}

